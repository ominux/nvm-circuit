\vspace{10pt}
\section{Education Plan}

 An academic job is not only to create knowledge but also to
 disseminate that knowledge to others both at the graduate and
 undergraduate levels. Education is an integral part of the PI's
 career development plan and is a supportive and inseparable part
 of the PI's research.  It is the PI's belief that research and academic
 activities should be inextricably linked in a healthy academic environment.
 The PI envisions that students play an important role in the
 research program. Successful teaching and other forms of interaction
 with students will attract them to the research area, excite them to
 learn more about the ongoing research, and eventually contribute to
 the PI's research program. The interaction between academia and
 industry is also very important.
 The following sections
present the objectives of the proposed educational plan.

\vspace{10pt}
\subsection{Course Development and Teaching}

\squishlist

\item {\bf Course module development}

As computer engineering educators, we should not only preserve the
historical domain of our discipline, but also expand it. Current
standard curricula on VLSI design/computer architecture/embedded
system design are still mainly oriented to {\it deterministic
design paradigm}, giving none or little emphasis on the
non-deterministic behaviors introduced by technology scaling or
novel device. In stead of introducing a new course, {\bf the major
goal of the PI is to develop and disseminate course modules that
complement or upgrade existing core courses, by introducing new
development and challenges in the forthcoming probabilistic design
paradigm}. These class modules (overheads, labs, and notes) on
process variations and probabilistic design techniques, will
easily upgrade and complement a variety of courses, including
embedded system designs, computer architecture, VLSI circuits and
systems, and design automation tools/methodologies. The course
modules and class projects related to PVT variations will be
developed to provide students with hands-on experience on
statistical timing analysis and statistical optimization
techniques. Some of the modules will be incorporated into
undergraduate engineering courses and others will be more suitable
for graduate-level courses:
    \squishlist
      \item {\it Undergraduate courses}. The PI plans to revamp a
      senior-level course (CSE 477: VLSI Digital Circuits) by
      introducing new course modules, such as circuit-level process
      variation, new labs on SPICE simulation of process
      variation, and gate-level statistical timing analysis tools.
      The PI also plans to incorporate variation-aware
      micro-architecture design concepts, such as the Razor
      architecture~\cite{PV:razor}, into a senior-level computer
      architecture course (CSE 431: Computer Organization and
      Design).
      \item{\it Graduate courses}. The PI introduced a new
      graduate level course (CSE 598C: Design of Reliable Power
      Efficient Systems) in his first semester on the faculty at
      Penn State.
      The new course already included the reliability
      issues caused by temperature variation and power supply
      noise variation.  Another graduate-level course that the PI
      co-teaches is a research seminar course (CSE597D: Embedded
      System Design).
      The PI plans to develop and incorporate new modules on
      process variation as well as the new research outcomes from
      the CAREER research program for these courses, and make them available to
      other academic institutions through the WWW or CDs, so that
      they can be easily integrated into a variety of courses. In fact,
      some of the modules from
      the PI's course have been adopted by University of
      Minnesota (Prof. Antonia Zhai) and University of Connecticut
      (Prof. Yunsi Fei). The PI believes that the new modules will
      continue to
      benefit his colleagues at Penn State and other institutions.
      \squishend

\item{\bf Fostering Interaction with Industry}

In teaching computer engineering courses, the PI believes in the
clear need to present real-life example products and applications,
so that students can understand the significance of what they are
learning. For instance, in the past, the PI infused his industrial
design experience into the classroom and discussed the real
problems he encountered when working as a designer. In this
education plan, the PI plans to invite his industry contacts, who
are the experts in process variation (Dr. Kerry Bernstein from IBM
and Dr. Tanay Karnik from Intel), to give guest lectures on how
complex chips are designed and what biggest challenges the
industry faces in the probabilistic design paradigm.

\item{\bf Bringing Research into the Classroom}.

The PI views research and teaching as complementary and promotes
this view in all the classes he teaches. For example, some
projects from his graduate course involve both theory and
implementation on new topics, and have led to conference quality
papers~\cite{xie:islped04,xie:isvlsi05, xie:asic05}. The PI will
propose related research topics as possible projects and the tools
developed via the research plan will also be used in the
classroom.



 The semiconductor industry is a fast growing
and swiftly changing area. The PI believes that it is extremely
important to keep the advanced-level graduate courses up-to-date
with latest research papers and novel ideas. When the PI was at IBM,
the company had a tradition of giving a review seminar to all
employees after a major conference (such as DAC or ISSCC). The PI
plans to borrow this idea and develop a graduate seminar course, in
which the students will review the latest major design automation
conferences and real-time embedded system conferences, such as DAC,
ICCAD, and RTSS. The PI will guide the students to review the most
significant papers from each conference, encouraging students to
explore their interesting topics and discover potential research
topics.

\item{\bf Striving for Teaching Excellence}

The PI has worked very hard to improve his teaching effectiveness
rapidly. He has consulted senior professors with numerous
questions about teaching and signed up for the teaching seminars
organized by the Penn State's graduate school. These efforts have
been successful; for the first 3 years of teaching at Penn State,
the PI received an average 6.03/7 rating for his teaching
evaluation, which was higher than the departmental and college
average (5.17/7 and 5.28/7, respectively). The PI will keep using
all resource available at PennState to improve his teaching. In
particular, he will keep participating all activities provided by
PennState's Schreyer Institute for Teaching Excellence
(www.schreyerinstitute.psu.edu), including teaching luncheon,
seminars, and workshops.

\squishend

\vspace{10pt}
\subsection{
Multidisciplinary/Multi-institution/International Education
Collaboration}

 The PI plans to investigate {\it collaborative teaching experiments} that
can then be adopted by other universities. In fact, The PI has {\bf
already} made a plan with Carnegie Mellon University and University
of Pittsburgh: The PI will work with {\it Prof. Rob Rutenbar} from
CMU and {\it Prof. Alex Jones} from UPitt  to organize a
graduate-level course on design automation tools and algorithms in
Fall 2006 (CSE 578: CAD Tools). This course will span design
automation flow from high-level synthesis to physical synthesis. The
course will be simultaneously offered at Penn State, CMU, and
University of Pittsburgh through online course delivery system
(WebEx). Lectures will originate from the different schools based on
the topics to be covered.

Based on the experience and assessment from this design automation
course, the PI plans to organize another graduate-level course on
probabilistic design flow for MPSoC embedded system design, which
complements the CAREER research plan. The course will involve
Princeton University (Prof. Wayne Wolf on conventional embedded
MPSoC design), North Carolina State University (Prof. Frank
Mueller on embedded software and compiler design), and
Northwestern University (Prof. Hai Zhou on statistical timing
analysis and optimization). The PI will incorporate the latest
research outcomes from this project. Students at
PSU/Princeton/Northwestern/NCSU will also experiment with the
tools developed as a part of this research.

This multi-institution education plan will not only provide a
unique opportunity for students to learn from experts in other
universities/areas but also promote collaborations among students
in different schools by working together on course projects. Such
remote collaboration is a critical skill in today's global
economy, where many companies have offices throughout the world.

The PI also believes that the success of universities in the
United States stems from their willingness to leverage the best
talent around the world. The PI plans to foster the connections
between PennState and other top universities in the world. In
fact, the PI will spend 7 weeks during the summer of 2006 visiting
top universities in Asia. He will spend 4 weeks at Tsinghua
University in Beijing to teach a short course on embedded system
design. He will then spend 2 weeks and 1 week at National Taiwan
University and Hong Kong University of Science and Technology,
respectively, to give seminars and explore research collaborations
with these universities. He also plans to conduct an international
teaching experiment with Tsinghua University by offering his
graduate level course on embedded system designs to Tsinghua
University's graduate curriculum.

\vspace{10pt}
\subsection{Outreach and Broader Impact}

{\bf Tutorials and Workshops:} The PI also believes that it is
important to share research findings with experts in related areas.
Such activities can spawn inter-disciplinary and inter-university
research collaborations. The PI has a good track record of
presenting tutorials on his latest research outcome, together with
other experts in the field. For example, \textbf{he has presented
tutorials} in ASPLOS 2004~\cite{xie:tutorial-asplos04}, ASPDAC
2005~\cite{xie:tutorial-aspdac05}, ISCA
2005~\cite{xie:tutorial-isca05}, ASICON
2005~\cite{xie:tutorial-asic05}, and will present in MICRO
2006~\cite{xie:tutorial-micro06}.  From past offerings of tutorials,
the PI has found these tutorials to serve as a spark plug for
drawing more researchers to start working on a particular research
area by creating awareness of the problem's importance and by
introducing tools to facilitate in solving the problem. The PI will
continue this tradition in this program, and prepare
\textbf{tutorials for the future embedded system conferences such as
EMSOFT, ISSS+CODES, RTSS, or CASES}, based on the new outcomes from
this research. The PI has served as the tutorial chair for
\textbf{EMSOFT 2005} and helped organized the conference. He also
served in the \textbf{CASES 2006} program committee. In the future,
the PI plans to organize a \textbf{workshop on probabilistic
embedded system design}, since a workshop is a great aid to attract
other researchers to exchange ideas in this important area.


{\bf Industrial Courses:} University and industry
     interactions are very important and benefit each other.
     On the one hand, the class materials in university education
     need to be infused with the latest developments in the industry;
     on the other hand, professional engineers need continuing
     education and training to develop new specialized skills and
     keep up with the rapid changes in the industry. Penn State is
     part of the University-Industry-Government partnership called
     {\it The Technology Collaborative (TTC)} (www.techcollaborative.org),
     which focuses on research,
     training, and education issues related with system design.
     The PI is actively involved with their education programs
     and has offered courses to the local industry in the past
     through TTC. Based on his graduate level course (Design of
     Reliable Power Efficient Systems), the PI has developed a
     two-day short course on designing reliable power efficient
     systems for industrial engineers. The course was delivered
     twice (in Jan 2004 and May 2004) to industrial engineers
     in companies like IBM, Seagate, and ADC, via an online
     course delivery system (Webex) as well
     as local attendance.  He was also invited by Synopsys Inc.
      to give a five-day short course to industrial engineers
      on designing reliable power efficient circuits.
      The PI plans to maintain a close relationship with
      industry and disseminate findings of the proposed
       research to industry practitioners, who in turn can
       incorporate these into real Embedded SoC designs.

\vspace{10pt}
\subsection{Student Advising}

%The proposed project will support two graduate students.
    %He also advises the senior undergraduate student on the thesis.

\squishlist

\item{\bf Advice Webpage}. Currently the PI has four Ph.D.
advisees and co-advisees, including one female student. The PI has
created an advice webpage (www.cse.psu.edu/\~yuanxie/advice.htm),
which is a collection of more than 100 useful links categorized as
follows: (1) Ph.D. dissertation/research advice; (2) presentation
advice; (3) technical writing advice; (4) technical reviewing
advice; (5) Job hunting advice; and (6) English learning advice.
These advice links were collected by the PI when he was a graduate
student at Princeton University and was extremely helpful for the
smooth completion of his Ph.D. study. The PI's advisees have found
them very useful when they start their graduate study. The webpage
has also benefited other faculty members' advisees and the PI has
received much positive feedback.

\item{\bf Advising Under-represented Groups and Undergraduate
Research}

The Computer Science and Engineering disciplines have exhibited a
growing gender gap~\cite{es:women-engineer1,es:women-engineer2}.
Working close together with his mentor Dr. Mary Jane Irwin, who is
very active in CRA-W and ACM-W, the PI has strived for the
promotion of women and minority in engineering. Currently he has
one female Ph.D. student, who has been involved in the preliminary
work proposed in this program~\cite{xie:iccad06}, and will be
supported by this program if it is funded. The PI has encouraged
her to participate PennState Women in Engineering Program (WEP),
to attend CRA-W (Women in Computing Research Association)'s
computer architecture summer school in 2006, and to attend DAC
2006 with ACM-W scholarship.  He plans to recruit two minority
undergraduate students through the Penn State WISER (Women in
Science and Engineering Research) and MURE (Minority Undergraduate
Research Experience) programs, with funding provided by Penn
State. These two students can help develop software for this
program. PennState's WEP program also organizes Engineering Camp
for Girls and Girl Scout Saturdays every year. The PI plans to
participate and design small projects for these students, to
promote computer engineering among high school female students.


 \squishend

%\vspace{10pt}
%\subsection{Community Service}





%{\bf Research Community Webpages.} It is the PI's belief that the
%advisor should also play an important role to help advisees get a
%big picture of the whole research community, and keep track of
%others' latest research outcomes. The PI has created a computer
%engineering academic genealogy
%(www.cse.psu.edu/\~yuanxie/community/genealogy/), which includes
%more than 600 researchers in VLSI/Architecture/Embedded
%System/Compiler areas. The PI also maintains a "Who is who" web
%page (www.cse.psu.edu/\~yuanxie/community/whoiswho/), with more
%than 1000 researchers' web links, research areas, DBLP entries,
%and photos. The two research community webpages have helped my
%advisees get an overview of the community. Actually, these two
%webpages have generated a lot interests from other researchers and
%were regarded as a good service to the overall computer
%engineering research community. These two pages by no means cover
%all researchers in the computer engineering community. The PI
%plans to keep updating the database and make it a useful tool for
%other researchers.

%{\bf Embedded Systems Week}



\vspace{10pt}
\subsection{Outreach and Broader Impact}

{\bf Tutorials and Workshops:} Li believes that the communication between academia and industry is very important. In NANOARCH 2009, she organized a panel on \textbf{Emerging Technologies}, which brought industrial voices into emerging NVM research. She also gave a tutorial in Tsinghua University in 2008. Li will continue her effort, and prepare tutorials for the future device and circuit conferences such as DATE, ISQED, or DAC based on the new outcomes from this research.  
