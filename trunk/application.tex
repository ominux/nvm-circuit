\section{Task 3: Novel Applications}

\subsection{Memristor as novel sensing scheme}
The device structure for temperature sensor is in Fig. 3a. It consists of a long spin-valve strip which includes two ferromagnetic layers: reference layer and free layer. The magnetization direction of reference layer is fixed by coupling to a pinned magnetic reference layer. The free layer is divided by a domain-wall into two segments that have opposite magnetization directions to each other. The device time domain resistance depends upon domain wall position as:  , where  and  are the high and low resistance of the spin valve.   is the spin valve length and   is the domain wall position.

Domain wall velocity at finite temperature depends upon both spin torque excitation strength and thermal fluctuation magnitude. Fig. 4 shows the normalized domain wall velocity as a function of the normalized current density for different normalized thermal fluctuation magnitudes. Domain wall velocity increases as temperature increases. Temperature sensitive and insensitive regions can be observed. Curves with kneeling shapes are around critical current density, where the domain wall velocity is sensitive to thermal fluctuation magnitude. For temperature sensing, a biasing voltage pulse with constant magnitude is applied to the device.  Resistance difference before and after voltage pulse is measured. This resistance difference is calibrated to sense temperature. Higher temperature results a bigger resistance dropping as shown in Figure 5.

The temperature sensing memristor is operated at a region where its electric behavior is sensitive to temperature change. This is achieved through a combination of temperature dependent domain wall mobility and the positive feedback between resistance and driving strength in memristor. The positive feedback between resistance and driving strength is a unique property of the memrisor. Memristor's resistance depends upon the integration of current/voltage excitation. For a constant voltage pulse driving, higher temperature results an increased domain wall moving distance. The increased domain wall moving distance results a smaller resistance. The smaller resistance results a higher driving current density, thus providing positive feedback to further increase domain wall distance. This positive feedback accelerates domain wall speed and reduces device resistance further for a constant voltage pulse driving. Solid curves on the Fig. 5 are the resistance changes for the proposed spintronic memristor at different temperatures. Dash lines are the resistance dropping for a non-memristive device without positive feedback between resistance and the integration of driving strength.  The dash lines are equivalent to simulations with fixed driving current strength. It can be seen that positive feed back between resistance and driving strength in memristor significantly increases the temperature sensing margin.

\subsection{Reconfigurable System}

\subsection{: Hybrid system with other emerging devices.}


