%\subsection{Device Modeling and Design Flow}
\section{Task 1: Design Methodologies for Emerging Memory Technologies}

This proposed task focuses on device modeling and design flow and optimization
methodologies for memory design using emerging memory technologies.
%\textbf{\underline{HL: Need to be modified.}}
%To help the architectural level and system-level design of
%the SRAM-based or DRAM-based cache and memory, various
%modeling tools have been developed during the last decade. For
%example, CACTI~\cite{CACTI:NORM,PRAM:EVANS,PRAM:eCACTI,PRAM:CACTI60} and
%DRAMsim~\cite{DRAMsim} have become widely used in the
%computer architecture community to estimate the speed, power, and
%area parameters of SRAM and DRAM caches and main memory.
%Similarly, to explore new design opportunities that these emerging
%memory technologies can bring to the designers at architecture and
%system levels, it is imperative to have a high-level model
%for caches and memories built with emerging NVMs, such as MRAM/PCRAM.
%The model needs to provide the extraction of all important parameters, including access latency, dynamic access power, leakage power, die area, and I/O bandwidth \emph{etc.}, to facilitate architecture and system-level analysis and to bridge the gap between the abundant research activities at process and device levels and the lack of a high-level cache and memory model for emerging NVMs.




\subsection{Task 1.1: Device Modeling for Emerging Memory Technology}

Not like SRAM which is based on traditional CMOS technology, new materials are introduced in the emerging NVM technologies. For example, MRAM arises from magnetic tunneling junction (MTJ), and PCRAM technology is based on Ge$_2$--Sb$_2$--Te$_5$.
Due to the lack of knowledge on material physics of these NVM devices, most of research works on circuit, architecture and system levels nowadays are based on highly-simplified characteristics of the emerging devices. This methodology can cause a large design overhead, increase the production cost, and reduce the design margin, especially in the highly scaled technology with large process variations.  For example, the data storage element MTJ at a certain resistance state is usually modeled as a constant resistor by ignoring the dependency of the MTJ resistance on the magnitude of the read/write current driven by the NMOS selection transistor in an MRAM cell. Our previous work~\cite{Chen08} showed that after adopting a dynamic MTJ model that can take into account the time-varying electrical inputs in MRAM design flow, the design pessimism can be dramatically minimized and the memory array area can be reduced by more than 40\%. Therefore, one of the important tasks of our proposal is to build device models of the emerging NVM technologies for circuit design. Both dedicated device model and simplified behavioral model will be developed.

The dedicated device models, which will be built based on physical mechanism and corroborated by device measurements, need to satisfy three requirements: (1) These models should provide not only the accurate static characteristics (i.e., I-V relationship and high/low resistances), but also the reasonable dynamic behaviors, for example, what is the relationship between write current amplitude and write current pulse width and frequency in PCRAM design? How does MTJ resistance change during the magnetic direction transition of ferromagnetic layer? (2) The device parameter fluctuations induced by process variations, such as line-edge roughnesses (LERs), oxide thickness fluctuations (OTFs), and random discrete dopants (RDDs), will be also analyzed and integrated into the dedicated device model; and (3) the models should have reasonable runtime and be compatible to commercial EDA tools, i.e., HSPICE from Synopsys~\cite{synopsys} and Spectre from Cadence. Hence, Verilog-A or C language could be used to implement these models. The dedicated model will be used for memory optimization and timing/power analysis.

On top of it, the simplified behavioral models will be extracted. High-level languages, i.e. VHDL/Verilog or C will be used. The highly simplified conceptual model will be used for logic and functionality analysis.

\subsection{Task 1.2: Memory Circuit Design Flow}

\begin{wrapfigure}{r}{0.6\textwidth}\centering \centering   \includegraphics[width=0.6\textwidth]{./figure/HL-flow.pdf}
\caption{The proposed scope of device modeling and circuit analysis methodology for the emerging NVMs.}\label{flow}  \vspace{-20pt}
\end{wrapfigure}

Another important task of our proposal is to build a design environment that can be seamlessly integrated with the existing CMOS logic design flow. \textbf{\underline{HL: Modify figure}}. Figure~\ref{flow} illustrates the proposed scope of device modeling and circuit analysis methodology for the emerging NVMs. In Stage I, we will develop the dedicated device models be based on physical mechanism. On top of it, the simplified behavior models will be extracted. In Stage II, we will build an emerging memory design flow, which can realize the creation and optimization of novel hierarchal memory array structure and peripheral circuitry. The accuracy of the corresponding device model will determine the credibility of the design, such as critical timing/power simulation and corner analysis. Therefore, the dedicated device model will be used in this step. High-level synthesis and function verification will also be an important part in Stage II. The simplified conceptual model is expected to provide sufficient accuracy and %for logic and functionality analysis.
can be easily integrated in the commercial EDA tools and design methodologies such as \emph{Primetime} and \emph{Timemill} from Synopsys~\cite{synopsys} for more thoroughly analysis, \emph{i.e.}, the critical path timing at design corners. In Stage III, we will build IP's (Intelligence Properties) for emerging NVM technologies with the aid of the proposed design flow in Stage II. The IP's will provide the extracted parameters of memory array cell including area, dynamic and leakage power, access latency, \emph{etc.}, the recommendable memory array structures and the corresponding trade-offs, as well as the optimized peripheral circuitry design, \emph{i.e.}, sense amplifier and write drivers. Those IP's will be used in the researches at architectural and system levels.

The whole methodology and the corresponding outcomes, including device models, memory design flow, and IP's, will be distributed to the architecture and system design community. Our project will build a channel and provide a friendly interface among material development, device fabrication and architecture design.

%=========================================================================
%\textbf{I removed the Architectural modeling portion, since it is in 
%our other proposal, reviewers could easily compare and claim that we 
%duplicate the writing and decline both proposals.}

\begin{comment}
\subsubsection{Architectural Modeling}

Based on the device/circuit-level modeling and analysis methodologies described in Task 1-A, we will develop a PCRAM/MRAM simulator, which can be easily integrated with architecture simulators including Simplescalar-based single core simulator~\cite{simplescalar:computer02,sim-alpha}, and multi-core simulators such as M5~\cite{3D:M5},GEMS~\cite{martin:can05} or PTLsim~\cite{PTLsim}.

Note that tools such as CACTI~\cite{CACTI:NORM,PRAM:EVANS,PRAM:eCACTI,PRAM:CACTI60} and DRAMsim~\cite{DRAMsim} have been widely used in the computer architecture community to estimate the speed, power, and area parameters of the traditional caches and main memory. However, these existing tools were initiated and built based on the cache and memory modelings of SRAM/DRAM. The architectural modeling for PCRAM/MRAM raises unique research issues and challenges on building such simulators. First, some circuitry modules in PCRAM/MRAM have different requirements from those originally designed for SRAM/DRAM. For example, the existing sense amplifier model in CACTI~\cite{CACTI:NORM,PRAM:EVANS,PRAM:eCACTI,PRAM:CACTI60} and DRAMsim~\cite{DRAMsim} is voltage-mode sensing, while PCRAM data reading usually uses a current-mode sense amplifier. Second, due to the unique device mechanisms, the models of PCRAM/MRAM need specialized circuits to properly handle their operations. We can still take PCRAM as an example. The specific pulse shapes are required to heat up GST material quickly and to cool it down gradually during the RESET and especially SET operations. Hence, a model of the slow quench pulse shaper need to be created. Finally, the most obvious and important difference between PCRAM/MRAM and SRAM/DRAM is their distinct memory cell structure. PCRAM and MRAM typically use a simple ``1T1R'' (one-transistor-one-resistor) or ``1D1R'' (one-diode-one-resistor) structure, while SRAM and DRAM cell has a conventional ``6T'' structure and ``1T1C'' (one-transistor-one-capacitor) structure, respectively. The difference of cell structures directly leads to different cell sizes and array structures.

In addition, where to place these NVM memories in the traditional memory hierarchy also influences the modeling methodologies. For example, the emerging NVMs could be used as a replacement for on-chip cache or for off-chip DIMM (dual in-line memory module). Obviously, the performance/power of on-chip cache and off-chip DIMM would be quite different: When a NVM is integrated with logics on the same die, there is no off-chip pin limitation so that the interface between NVM and logic can be re-designed to provide a much higher bandwidth. Furthermore, off-chip memory is not affected by the thermal profile of the microprocessor core while the on-chip cache is affected by the heat dissipation from the hot cores. While higher on-chip temperature has a negative impact on SRAM/DRAM memory, it actually has a positive influence on PCRAM because the heat can facilitate the write operations of PCRAM cell. The performance estimation of PCRAM becomes much more complicated in such a case.
Moreover, building an accurate PCRAM/MRAM simulator needs close collaborations with the industry (see collaboration letters from HP, IBM, IMEC, and Seagate) to understand physics and circuit details, as well as architectural level requirements such as the interface/interconnect with the multi-core CPUs.

\end{comment}


\subsection{Task 1.3: Energy/Performance/Reliability Design Space Exploration}

The physical characters of a NVM cell is mainly depends
on the material characters and fabrication process. However, the circuit design
of the memory array, such as the access device of the memory cell, the operational
voltage, and the pheriphral circuitry, can also impact the operation
conditions of the cell, such as current and power consumption. In this task, we propose to explore the design space of the NVM memory array, and study the energy/performance/reliability tradeoffs for memory design with such emerging
memory technologies.  

Due to the intrinsic non-volatile characteristics of these emerging memory technology, naturally the read and write behaviors are asymmetric in terms of performance and energy. For example, the write-operation of PCRAM/MRAM requires a large current to be applied for a period of time so that the state of the storage junction is flipped; while the read-operation is realized by applying a small voltage to the cell and sensing the current across the cell.
We propose to analyze the constrained conditions of he design of NVM memory array, and study the sizing of the transistors as well as the operational voltages to investigate the tradeoff of the distinguishability~\footnote{During the read operation, the ratio of high resistance to low resistance in the storage junction reflects the distinguishability between logic 1 and 0}, energy consumption, lifetime and speed. For example, both the width of NMOS access device and word-line 
voltage can obviously affect the distinguishability. A higher word-line voltage or a larger NMOS device is more desirable to obtain a high distinguishability. However, the power consumption issues and area consideration always require a low voltage and small device size. 
Another example is on the lifetime/energy/performance tradeoff. 
For example, the lifetime of PCRAM is represented by the cycling
endurance, which is a function of pulse energy applied for 
the memory cell during the RESET writing. The reason for 
higher energy pulse induced cycle lifetime degradation 
is that the RESET resistance can be saturated when the 
writing current is higher than a critical level. This
"over programming`` phenomena can result in lager amorphous
volume, and then degrade the PCRAM��s lifetime. Consequently, 
a large write current can help improve the performance, but affect both lifetime and energy. Consequently, depending on the application,
we plan to investigate two optimization strategies: (1) \textit{Energy-driven optimization.} For low power application, such as mobile computing platform, energy consumption may be the most important design goals. We can optimize the word-line and bit-line voltage as well as the transistor sizing of the access NMOS device for memory array to achieve minimal energy consumption while satisfy constrains on lifetime, performance, and area. (2) \textit{Performance-driven optimization.} For high performance application, We can perform the optimization to achieve the best read/write performance while satisfy constrains on lifetime, energy, and area. Such optimization strategies can also be extended to lifetime-driven optimization and density-driven optimization.  


\subsection{Preliminary Result and Collaborations:}
The PI Li has built a combined magnetic and circuit design analysis and optimization methodology for MRAM, which has been proved to improve design efficiency significantly~\cite{Chen08} by test-chip design and fabrication at Seagate. We are also one of the first researchers to propose spintronic memristor structures~\cite{Wang09}, which was interviewed by IEEE Spectrum~\cite{Spectrum09}. The corresponding compact model and corner analysis~\cite{Chen09} have also been developed. In this project, we will further extend this methodology to other emerging NVMs, such as PCRAM.

The PSU PI Xie has developed a stacked SRAM cache
simulator called 3DCacti~\cite{xie:iccd05-3d, XIE:TVLSI2008-3DCacti},
which has been widely downloaded and used by other researchers.
The PI and co-PI have collaborated together when the PI Li was in Seagate,
to develop a preliminary version of MRAM simulator for cache stacking~\cite{Xie:dac08,XIE:HPCA09}.
Xie also collaborated with Dr. Norm Jouppi
from HP Labs, developed a preliminary version of PCRAM simulator~\cite{xie:pcramsim}.
We will extend our existing toolsets to support architectural exploration.
 
