
\section{Related Work}
\label{related}

In recently years, there have been active efforts on emerging NVM technologies. However, most of efforts were at process and device levels. Relatively, the architecture and system level analysis is less due to the lack of a high-level cache and memory model for emerging NVMs.

\paragraph{PCRAM.}
%%%% Device level
Compared to STT-RAM, PCM is even denser with an approximate cell area of $6\sim12F^2$~\cite{ITRS07}, where F is the feature size. In addition, phase change material has a key advantage of the excellent scalability within current CMOS fabrication methodology~\cite{Cho05,Kim06,Lai01,Pirovano03,Raoux08}.
Continue density improvement is the most important task for PCRAM process development. 2 and 4-bit MLC PCRAM material and the correponding write strategies were demonstrated by Nirshl et al.~\cite{Nirshl07}. New process integration techniques, such as ultra-small lithography independent contact area~\cite{Chen07-iedm} and unified 7.5nm dash-type confined cell~\cite{Im08}, could also help enhance density. Reliability is the major challenge in PCRAM process, which severely limits its applications. Researches on different angles have been done: Lacaita et al. discussed projected its impact on scaling ~\cite{Lacaita07}, Shih et al described the mechanisms of retention loss in GST material~\cite{Shih08}, and Lavizzari et al presented the impact of transient effects~\cite{Lavizzari08}. Accordingly, many device models were built from reliability~\cite{Ielmini07}, low-frequency noise~\cite{Fantini08}, statistical analysis~\cite{Mantegazza07} point of views. Those models mainly were dedicated to process and device, which cannot be borrowed by computer community.

%%%% circuit prototype to approve the potentials PCM
Many PCRAM prototypes have been demonstrated in the past years. In 2007, a 1.5V 512KB embedded PCRAM in a 0.13$\mu$m CMOS by Hitachi~\cite{Hanzawa07} and a 512b diode-switch PCRAM in a 90nm CMOS by Samsung~\cite{Lee07-isscc}. A year later, a 256Mb MLC PCRAM in a 90nm technology by STMicroelectronics~\cite{Bedeschi08}. Accordingly, a multi-level programming algorithm was developed and embedded into the chip, demonstrating 2b/cell feasibility. Very recently, a 45nm 1Gb 1.8V single-level cell (SLC) PCRAM was designed with 85ns random-access time and 9MB/s program throughput was demonstrated by Numonyx~\cite{Villa10}, and A 90nm 4Mb embedded PCRAM with 1.2V 12ns read access time and 1MB/s write throughput was made by STMicroelectronics~\cite{Sandre10}. In addition, The peripheral circuit design for high density diode-switch PCRAM was discussed in ISQED 2009~\cite{Li09}. Diode-switch PCRAM was demonstrated in VLSI Symposium 2007~\cite{Zhang07}

%%%% arch level for PCM

Discussions on endurance limitation were recently brought up to off-ship memories. Lee et al. proposed techniques to reduce the write accesses to PCM-based main memory to improve its endurance~\cite{ISCA09+MSPRAM}. One of the techniques utilizes the dirty bits in L2 at the word granularity to check if a word has been updated since it was last fetched on-chip. A technique along a similar line for PCM memory at bit level was proposed by Zhou et al.~\cite{ISCA09+Yang}. In this technique, a memory row is first read out, then compared with the new data, and finally written back for those changed values. This technique can significantly save performance and energy by avoiding pre-write operations. These technologies, however, all incurred significant access performance degradation and cannot be directly used in on-chip cache structure.


\paragraph{MRAM.}
%%% device
MRAM features non-volatility, fast access speed, zero standby power and high programming endurance\cite{Hosomi05,Diao07}. The research on MRAM material and device mainly devoted on spin-torque based MTJ due to its better performance, higher density, and better scalability compared to the conventional MRAM~\cite{Kawahara07,Salahuddin07,Beach08,Kishi08}. Certainly, the yield improvement is an important topic in nanoscale devices. Miura et al. presented a SPRAM with synthetic ferrimagnetic free layer, which has high immunity to read disturbance and sufficient margin between read and write currents~\cite{Miura07}. The MTJ structure with synthetic ferrimagnetic free layer can achieve a lower critical current density without degrading the thermal stability~\cite{Durlam03}.
Very recently, a 2-bit MLC (Multi-level cell) MTJ device was reported in~\cite{Lou08} for further density enhancement. Two-digit information -- $00$, $01$, $10$, and $11$, are represented by four MTJ resistance states. The transitions between different MTJ resistance states can be realized by passing the spin-polarized currents with different amplitudes and/or directions.

%%%% circuit prototype to approve the potentials
A 4Kb STT-RAM using tailored MTJ design was fabricated by Sony in $0.18{\mu}m$ technology in 2005~\cite{Hosomi05}. The test chip demonstrated that STT-RAM is a prominent candidate for the next generation memory because of its high speed, low power and high scalability. In 2007, Kawahara et al. prototyped a larger 2Mb STT-RAM in $0.2{\mu}m$ technology~\cite{Kawahara07}. This chip improves memory access latency by featuring an array scheme with bit-by-bit bidirectional current write and a parallelizing-direction current read. Recently, a even larger capacity -- 32Mb MRAM prototype in 90nm technology was demonstrated by NEC~\cite{Nebashi09}. A cell structure with 2 transistors and 1 magnetic tunneling junction (2T1MTJ) was adopted to improve access time to 12ns. Besides the SRAM-like array~\cite{Motoyoshi04,Andre05,Kawahara08}, other memory structures are also investigated by using MRAM/STT-RAM technology. In~\cite{Wang07}, Wang et. al. described a CAM structure based on conventional MRAM technology. In~\cite{Xu08}, Wu et. al. propsoed a novel STT-RAM read scheme with high sensing margin and illustrates a new CAM design. The possibility of applying STT-RAM in reconfigurable logic block for 3D-stacked reconfigurable spin processor was investigated~\cite{Sekikawa08}.

A write disturbance fault (WDF) model for conventional MRAM was proposed by Su et al.~\cite{Su08}. The fault affects the data stored in MRAM cells due to excessive magnetic filed during a write operation.
This should not be a problem to STT-RAM since it uses spin-polarized current to flip data. We have proposed a dynamic MTJ model with more accurate (transient) description for MTJ resistance switching~\cite{Chen08}. Compared to highly conceptual fixed resistance used in traditional STT-RAM design flow, the dynamic model can help to reduce 20\% pessimism in write time at TSMC $0.13{\mu}m$. The failure probability of STT-RAM cells due to parameter variations was considered and discussed in~\cite{Li09}. A model was proposed to predict memory yield and design optimization to minimize memory failures.

%%%% arch level for STT-RAM
At architecture level, there are several recent efforts in using STT-RAM as an on-chip last level cache. Desikan et. al. conducted an architectural evaluation of on-chip conventional MRAM cache in a single micro-processor~\cite{Desikan02}. Dong et al. developed a delay and energy model for MRAM-based cache and conducted a detailed comparison between the cache with SRAM and STT-RAM technologies in terms of area, performance and energy in the context of 3D stacking~\cite{MRAM:DONG08}. Sun et al. extended the application of STT-RAM based cache to Chip Multiprocessor (CMP) and proposed new techniques to improve latency and to reduce write energy~\cite{Sun09}.

