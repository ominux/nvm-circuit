
\section{Broader Impacts, Outreach, and Education}

\paragraph{\textbf{Research Impact and Technical Merit:}}
Memory hierarchy design is one of the key components in modern computer
systems. The importance of the memory hierarchy
increases with the advances in performance of the
microprocessors~\cite{ITRS07}.  A key \textit{transformative aspect} of the proposed research is
that the success of the project will result in innovations in the computer architecture,
potentially leading to better performance, higher energy-efficient, and more reliable
computer systems.


\paragraph{\textbf{Collaborations and Partnership:}} It is naturally important to have industry support and guidance for this research.  The NYU-Poly PI Li has been with industry for 5 years before joining academia.  She has a strong connection with Memory
Product Group at Seagate, where she did research and led a design team on nonvolatile memories.    The PSU PI Xie worked for IBM Microelectronics division
before joining academia, and has built a good relationship with IBM
research. In the past 6 years as a faculty member, Xie has close collaborations with industry
partners. The proposed research has intrigued our industry partners, and the project will be
carried out with close collaboration with partners in several
companies, including IBM, Intel, HP, IMEC, Qualcomm, and Seagate.
%The industrial collaborators will
%play important roles in the proposed project by enabling the
%acquisition of realistic data, discussion of the practicality of
%ideas, placement of students in internships and permanent
%positions, and eventually the transfer of the technologies. By working
%closely with researchers in industry, the PIs will be able to ensure
%that the proposed methodologies and techniques are practical and have
%a real impact on industry.
The investigators
anticipate that the techniques and tools developed in this project will be
used in both classroom projects and academic/industrial
research. We will closely work with our industry partners to
transfer research results into commercial designs. The proposed
technology is of immense interest for companies.


\paragraph{\textbf{Outreach and Knowledge Dissemination:}} As part of outreach
efforts, the PIs will actively
disseminate results to a wide audience
and to different professional communities. The NYU-Poly PI Li believes that the communication between academia and industry
is very important. In NANOARCH 2009, she organized a panel on Emerging Technologies,
which brought industrial voices into emerging NVM research.
The PSU PI Xie has delivered over 30 invited talks in the
past at IEEE Chapters, universities, and companies.
He has been a tutorial speaker at several forums, offering
tutorials on 3D ICs in MICRO 2006, ISCA 2008, GLSVLSI 2008, and MICRO 2009~\cite{xie_url}.
Penn State is part of
the University-Industry-Government partnership called The
Technology Collaborative (TTC) that focuses on research, training
and education issues related with system design. The PI from Penn
State has been actively involved with their education programs and
have offered courses to the local industry in the past through
TTC. We will use this forum to disseminate findings of the
proposed research to industry practitioners, who in turn can
facilitate technology transition and incorporate research
breakthroughs in real systems.

\paragraph{\textbf{Women and Minority Student Recruiting Activities}}
While this research program will make contributions in educating all students to be well prepared for designing future computer systems, it will make additional efforts to promote diversity. Being a woman faculty herself, the NYU-Poly PI Li plans to actively recruit and mentor women and minority students. The PSU PI has an impressive record of graduate student advising, especially those from underrepresented groups, having graduated several women and minority graduate students. The PIs will continue to attract underrepresented students by getting their current graduate students from underrepresented communities to present their research at minority undergraduate institutions and to serve as role models. The PIs have been working with women and minority recruiting programs in both universities, i.e., the Multicultural Education and Programs at NYU-Poly and the WISER (Women in Science and Engineering Research) and MURE (Minority Undergraduate Research Experience) programs at PSU.


\paragraph{\textbf{Integration with Education:}} This project will
involve graduate and undergraduate students in all aspects of the
research. The PIs, as in the past, will actively integrate the
research results from this project into the graduate and
undergraduate curricula, especially related to computer architecture.
The NYU-Poly
PI teaches a graduate-level course EL5473 (Introduction to VLSI), and this
project will allow the PIs to integrate additional practical
material to make the class more appealing for engineering
students. A graduate-level course on advanced topics in computer architecture
will be developed at NYU-Poly in collaboration with colleagues who are
experts in architecture and circuit design. Undergraduate students
will be especially targeted and encouraged to pursue graduate
studies.  Support for undergraduate researchers will also be
sought from NSF REU supplements and  by involving
the outstanding students from the Schreyers Honors program at Penn
State.
Beyond involving students in all aspects of research, the PIs will
develop new courses on different aspects of advanced computer architecture and VLSI, to train the next generation work-force. In
addition, the PIs plans to organize workshops and tutorials at
major conferences to support other faculty to adapt new teaching
and research material in their curricula. Class notes, slides, and
laboratory manuals related to the new courses developed will be
made publicly available.
The PIs will educate industrial
practitioners and use this grant to disseminate findings to
industry practitioners, who in turn can facilitate technology
transition and incorporate research breakthroughs in real systems.


\paragraph{\textbf{Collaborative Teaching Experiments}:}  A graduate-level course
on emerging non-volatile memories will be simultaneously offered at Penn State and NYU-Poly (in a \textit{virtual classroom}) through an online course
delivery system (WebEx). Lectures will originate from both schools
based on the topics to be covered. The PIs will incorporate the
latest research outcomes from this project. Students at PSU and
NYU-Poly will also experiment with the tools developed as a part of
this research. This multi-institution education plan will not only
provide a unique opportunity for students to learn from experts in
other universities/areas but also promote collaborations among
students in different schools through working together on course
projects. Such remote collaboration is a critical skill in today's
global economy, where many companies have offices throughout the
world.

\paragraph{\textbf{Training of Students}:} Student mentoring is a key
component of this project. The PSU PI has excellent records in
student training.  His mentoring efforts were recently
  recognized with two Ph.D. students winning the department's ``Best Research Assisant Award" in 2008 and 2009. He has graduated 3 Ph.D. students (one in Sun
Microsystem, one in Qualcomm, and the other one in TSMC).
His students have received one best paper award (in ASPDAC
2008), and three best paper award nominations (in ICCAD 2006 and
ASPDAC 2009, 2010). The NYU-Poly PI just starts her academia career
recently in Fall 2009. She will be getting advice from the PSU PI on how to mentor and train graduate students during the course of this 3-year project.

